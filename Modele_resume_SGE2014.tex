% Mod�le pour la soumission des r�sum�s � SGE 2014
% http://sge2014.sciencesconf.org/
%
% Auteur : Romain Corcolle, LGEP (Laboratoire de G�nie Electrique de Paris)
% modifi� par : Pierre Haessig, SATIE
% Licence : document plac� dans le domaine public (CC0)
%
% Pour r�cup�rer la derni�re version du template :
% https://github.com/pierre-haessig/template-SGE2014
% Pour rapporter un probl�me :
% https://github.com/pierre-haessig/template-SGE2014/issues
\documentclass[10pt]{article}

\usepackage[french]{babel}
\usepackage{graphicx,epsf,times}
\usepackage[T1]{fontenc}
\usepackage{psfrag}
\usepackage{fullpage}
\usepackage{subfigure}

\makeatletter
\def\captionof#1#2{{\def\@captype{#1}#2}}
\makeatother

\setlength{\hoffset}{-5mm}
\setlength{\voffset}{-15mm}
\setlength{\textheight}{25cm}
\setlength{\textwidth}{17cm}


\title{
\vspace*{-0.5cm}
\begin{flushright}
\includegraphics[width=4.25cm]{logo_sge.png}\\
\normalsize{8-9 juillet 2014, Cachan}
\end{flushright}
\vspace*{36pt}
\fontsize{18}{18}\selectfont
\textbf{
Titre de la communication (Style Titre.principal : Times new roman, 18 pts, gras, centr�, interligne simple, 36 pts avant et apr�s)
}
\normalsize
\vspace*{12pt}}

\author{Pr�nom NOM des auteurs (Style Nom : Times new roman, 11 pts, centr�, interligne simple, 6 pts apr�s) \normalsize}

\date{\normalsize{Affiliation des auteurs (Style Affiliation : Times new roman, 10 pts, centr�, interligne simple, 12 pts apr�s)}}






\begin{document}

\maketitle

\thispagestyle{empty}

\noindent \textbf{RESUME -- �crire un r�sum� de la communication de 10 lignes maximum. Ce r�sum� doit pr�senter de fa�on synth�tique les objectifs du travail pr�sent�, les principaux r�sultats et insister sur les originalit�s du travail. (Style R�sum� : Times new roman, 10 pts, gras, justifi�, interligne simple, 12 pts avant, 12 pts apr�s)}\\[12pt]
\textbf{MOTS-CLES -- �crire ici une liste n'exc�dant pas 8 mots-cl�s significatifs. (Style Mots-Cl�s : Times new roman, 10 pts, gras, interligne simple, 12 pts avant, 30 pts apr�s).}

\vspace*{12pt}

\section{Introduction (Style Titre 1 : Times new roman, 14 pts, gras, justifi�, interligne simple, 12 pts avant, 6 pts apr�s, hi�rarchisation de 1er niveau de type 1.1.1\ldots)}

\noindent D�crire le contexte et les objectifs du travail. Positionner le travail par rapport � la litt�rature et aux principaux travaux ant�rieurs. Pr�senter le plan de la communication. (Style Normal : Times new roman, 10 pts, justifi�, interligne simple, 6 pts apr�s).


\section{Titre de section (Style Titre 1)}
\subsection{test}
\subsubsection{test2}

\begin{equation}
y=a.x+b
\label{eq_homog}
\end{equation}

\begin{figure}[!htbp]
\centering
\includegraphics[width=5cm]{logo_sge.png}
\caption{test}
\end{figure}


\thispagestyle{empty}

\begin{thebibliography}{99}

\bibitem{bib_BBG}{M. Bornert, T. Bretheau et P. Gilormini, \textit{Homogenization in Mechanics of Material}, Iste Publishing Company, 2007.}

\bibitem{bib_LDRC}{L. Daniel et R. Corcolle, ``A note on the effective magnetic permeability of polycrystals'', \textit{IEEE Trans. Magn.}, vol. 43, no. 7, pp. 3153--3158, 2007.}

\bibitem{bib_axell}{J. Axell, ``Bounds for field fluctuations in two-phase materials'', \textit{J. Appl. Phys.}, vol. 72, no. 4, pp. 1217--1220, 1992.}



\end{thebibliography}


\end{document}
